\chapter{Introduction}

Cardiac disease is one of the biggest causes of death in the UK, causing over
one third of all deaths.
In addition to the deaths, many more people suffer the after effects of a heart
attack or live with the difficulties caused by heart failure~\cite{bhf2008}.
These figures are duplicated across much of the developed world.


\section{Motivations}

The atria have generally been neglected in studies, compared to the ventricles
at least.
Many more studies focus on ventricular tissue than atrial tissue, and many of
the atrial studies focus more on the pacemaker of the heart, than the atria as a
whole.
The atria have a complex electrophysiology and topology.
Whilst atrial dysfunction is rarely fatal, atrial arrhythmias are amongst the
most common cardiac diseases, reducing the quality of life for hundreds of
thousands of people.
They have recently been the focus of a variety of clinical and physiological
interest, interest which has not yet been reflected in cardiac modelling.

Mathematical modelling of the heart offers a way of gaining insight into the
cardiac processes and the mechanisms of cardiac disease.
It is a well established field of research with numerous international journals
and conferences discussing the findings.
Mathematical models allow physiological effects to be dissected and quantified
in ways that can be difficult for in vivo and in vitro experiments.
This can be used to inform both further experiments and clinical diagnosis and
treatment.

Despite these benefits, mathematical modelling has a number of downsides.
One of these is the technical expertise needed to model the heart.
It is a non-trivial programming task to setup a computer to solve the equations
of a mathematical model.
There are existing toolkits which solve this issue, but they have limitations.
Once the model has been set up, it needs to be used in a variety of experimental
protocols, to quantify the behaviour of the model and any abnormal conditions
the experimenter is interested in.
This task is both complicated, as the experimental protocols can involve complex
pacing patterns and require detailed measurements to be taken.
The task is also quite simple and repetitive, in that the same protocols are
wanted to assess many different cell types and abnormal conditions.
Finally, such protocols can vary between experimenters, making it harder to
compare results between different studies.

A second problem is that of clinical relevance.
Many computational experiments focus on simplified models of cardiac tissue in
one or two dimensions.
Whilst such experiments are useful for elucidating complex interactions, they
can be of limited use to a clinician.
The clinical electrocardiologist typically works with external tools such as the
ECG.
Clinical procedures are more expensive, stressful and sometimes dangerous.
Diagnosis therefore depends on using the ECG to infer the activity within the
heart.
Being able to link the electrical activity within a model to the observed
surface ECG can help with this, allowing an atrial model to be used to test
hypotheses with direct clinical relevance.


\section{Aims}

To address these issues, the thesis has three broad aims.
To construct a toolkit suitable for modelling cardiac tissue and particularly
atrial cells, to construct a model of the human atrium and the surface ECG and to
use the toolkit and the model in studies of the human atrium.

The cardiac simulation toolkit aims to address several problems of cardiac
modelling.
As a toolkit, it simplifies the basic set up for numerical experiments by
providing mathematical models ready to solve.
In addition, it will provide a number of experimental protocols, reducing the
repetitive work required to perform numerical experiments on cardiac tissue and
making cross-study comparisons easier.
The toolkit will also take advantage of optimisation techniques, making such
studies faster to undertake.

The model of the human atrium should allow for the modelling of some of the
complexities of atrial tissue; this includes regions of heterogeneous
electrophysiology and anisotropic conduction.
Through use of appropriately biophysically detailed models as the basis, it will
be suitable for use in studies such as electrophysiological remodelling or
inherited gene mutation.
Optimisation techniques will be used to make the problem computational
tractable.
The atrial model will be coupled to a representation of the human torso and used
as the basis of a forward calculation of the surface P-wave ECG.
This will allow studies of direct clinical relevance.


The tools and models will then be used in studies of atrial electrophysiology.
These will involve modelling the atrium on a number of scales from the single
cell through to the P-wave ECG.
The studies will incorporate both normal electrophysiology and a variety of
abnormal electrophysiological conditions.

\section{Synopsis}

This thesis consists of seven chapters.
In these chapters, a general background of the subject is given, before more
specific details of the toolkit developed are given.
There is then a section of experimental work with single cell, 1D, 2D and 3D
models.
A model of the body surface potential is then developed before being used in
experimental studies of the P-wave ECG.
Finally, conclusions and future work are given.

\textbf{Chapter 1}: This chapter.
A introduction to the motivations and aims of the thesis and a summary of the
chapters contained within.

\textbf{Chapter 2}: The physiological and mathematical background needed to
understand this thesis.
A description of the heart, with emphasis on the atria, is given, from the micro to the macro scale.
The normal functioning of the heart is described, both on cellular and whole
organ levels.
Mathematical models of cardiac tissue on all scales are introduced.
This includes a brief history of model development and information on both the
benefits and limitations of modelling.

\textbf{Chapter 3}: The development and components of a cardiac simulation
toolkit.
A description of the technology and techniques which have gone into the
development of the cardiac simulation toolkit.
Details are given of the experimental protocols modelled by the toolkit.
The features offered by the toolkit are compared with the offerings of existing
toolkits.

\textbf{Chapter 4}: Experimental studies in the atrium.
The atrial model developed in the thesis is presented, along with validating
information.
Experimental studies are presented, using the toolkit and simplified
versions of the whole atrium model.
These studies include a familial gene mutation, atrial fibrillation induced
remodelling and a novel current found in the human atrium.

\textbf{Chapter 5}: The body surface potential or forward problem.
The mathematics of computing the body surface potential from the electrical
potentials in the heart are given along with implementation details of the
software used to solve them and the torso model used.
The effects of internal inhomogeneities in the torso on the generated ECG are
investigated.
The generated ECGs are compared with clinical data from both twelve lead and
body surface potential mapping.

\textbf{Chapter 6}: Applications of the forward problem.
The body surface potential model is used in two clinical studies.
The first uses the model to validate an existing algorithm for predicting the
origin of focal tachycardia based on clinical data.
The second uses the model to investigate the causes of a novel clinical
phenomena, inverted P-waves at night.

\textbf{Chapter 7}: Discussions and Conclusions
This includes a look to the future and the many avenues for future research
offered by the toolkit and models developed in the thesis.

