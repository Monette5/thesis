The University of Manchester\\
Jonathan David Stott\\
Doctor of Philosophy\\
Developing Realistic Models of the Human Atrium and the P-Wave ECG\\
23rd September 2009\\

Cardiac disease, including atrial fibrillation (AF), is one of the biggest causes of
morbidity and mortality in the UK, accounting for one third of all
deaths.\cite{foo}
Cardiac modelling is now a well established field.
Mathematical models offer a valuable way of gaining insight into the dynamic
behaviours of the heart, in normal and pathological conditions.
Great efforts have been put into modelling the ventricles, whilst the atria have
received less focus.
This thesis therefore concentrates on developing models of the atria.

In the first part of the thesis, I developed a simulation toolkit for
modelling myocyte electrophysiology and excitation waves in 1D \& 2D
tissues.
It includes optimisations such as adaptive stimulus protocols.
As examples of application, it is used to investigate effects of a novel anion
bearing current on atrial excitation and the effect of AF remodelling on atrial
myocyte electrical heterogeneity.

In the second part, a computationally efficient and anatomically based model of
the atria is constructed.
The 3D model includes heterogeneous, biophysically detailed
electrophysiology and conduction anisotropy.
The full model activates in \ms{121}\ in sinus rhythm, in
close agreement with clinical data.
The model is used, with the toolkit, to investigate the function effects of S140G
mutation in KCNQ1 which is associated with familial AF.

In the last half of the thesis, the 3D model forms the core of a boundary
element model of the P-wave Body Surface Potential (BSP).
The BSP model incorporates representions of the lungs and the heart blood masses.
Generated ECGs show qualitative agreement with clinical data.
Their morphology is as expected for a healthy person, with a lead II duration of
\ms{103}.
The BSP model is used to verify an existing algorithm for focal atrial
tachycardia location and in providing explanation for a novel clinical
phenomena, inverted P-waves at night.


Models of the human atria and body surface potential are constructed.
The models are validated against both experimental and clinical data.
These models are suitable to use as the platform for further research.


