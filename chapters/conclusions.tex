\chapter{Discussion and Conclusions}


This was an investigation into models of cardiac tissue, with a special focus on
atrial tissue.
It included whole atrial models, which were then extended beyond the heart to
simulate the P-wave ECG.
A toolkit providing access to efficient implementations of various experimental
protocols was developed.
A model of the atrium based on biophysically detailed myocyte models, including tissue heterogeneity and anisotropic
conduction, was constructed.
The atrial model was used as the basis of a model of the P-wave ECG which was
solved using a boundary elemental formulation.
All of the tools and models developed were used to perform physiological
studies of tissues in healthy and diseased states.

\section{Cardiac Simulation Toolkit}

The cardiac toolkit which was developed provides easy access to a wide variety
of experimental protocols.
These protocols are used to assess the physiological impact of a gene mutation,
drug action, hormonal effect or other electrophysiological modification.
They include protocols which act on single cells and also on one dimensional
idealisations of cardiac tissue.

The implementations of these protocols focused on efficient algorithms
which take advantage of the computational nature of the models.
It was important that this did not compromise the physiological accuracy and
level of detail employed.
This was accomplished in part through the use of lookup tables which reduced the
computational effort needed to solve \ms{1}\ of cardiac activity compared to an
implementation lacking such tables without impacting measured physiological
characteristics significantly.
Such a saving might be considered a `cell level' performance optimisation.
Greater savings can be observed through the use of `protocol level' performance
optimisation.
These protocol level optimisations are where the computational nature of the
models can really be exploited.
Optimisations include storing of cellular state after the pre-pacing part of the
protocol and adaptive stepping when tracking response curves of varying slope.
There was also the novel use of a basic computer science algorithm, the binary
search, to determine the limits in a number of experimental protocols.
This can be used, with a sensible choice of initial guesses, to reduce the total
number of cases which must be tested by an order of magnitude.

The toolkit also offers an easy way of specifying and running 2D simulations.
Simulation, even of irregular geometry, is simplified.
Utilities can be used to convert quantified colour bitmaps into simulation
geometries with heterogeneous electrophysiology.
A variety of stimulation protocols can be specified, including both current and
voltage stimuli.
The simulations are accompanied by regular outputting of gif snapshots of the
voltage, to allow simulations to be assessed as they run.
In addition, the tissue simulation code is parallelized using a shared memory
paradigm with OpenMP.

The toolkit was used to complete two computational studies.
These examined the influence of a novel anion bearing current, \ii{ANION}, in
atrial myocyte cells and the effects of atrial fibrillation induced
electrophysiological remodelling on the heterogeneity of atrial myocyte
electrophysiological properties.
The two studies represent different extremes.
In the \ii{ANION}\ study the influence on the action potential itself was very
small, although even this had significant influence on the dynamic behaviours.
In the remodelling study, the effects were obvious at all levels and this was
reflected in the persistent spiral wave activity.

\section{Whole Atrial Model}

The whole atrium model which was developed is capable of modelling the
electrophysiological behaviour of the atrium at the whole tissue level.
It includes both conduction anisotropy and tissue heterogeneity.
The use of biophysically detailed myocyte model as the basis makes the model
flexible.
Diffusion of voltage was altered to produce conduction velocities in the atrial
tissue comparable with those measured experimentally and to reproduce clinical
measurement of total activation time under sinus rhythm.
The model is based on an anatomical geometry, not an idealised one, including
fluctuations in wall thickness and other anatomical deviations.
All these features make the model suitable for modelling a variety of conditions.
This includes those induced by fundamental changes to the electrophysiology
caused by genetic defects or drug actions.

The level of detail might make the model computationally unattractive, as it
involves solving many ordinary differential equations to advance the cellular
state as well as calculating the diffusion between many cells.
The model is parallelized, using a shared memory paradigm, allowing the effort
to be split over several cores.
The biophysically detailed model used as the basis uses precomputed lookup
tables to calculate many of the parameters which influence the gating variables,
reducing repetition of effort at minimal cost of overall accuracy.
Repeated calculations are also reduced by pre-computation of the Laplacian
operator used to diffuse the voltage.
This is especially useful in the relatively thin walled atria, which therefore
have many more boundary cells.
All these refinements keep the total computational time within acceptable limits
whilst maintaining the biophysical detail required for many studies.

Using the toolkit and a simplified version of the whole atrial model the
effects of a novel gene mutation, the S140G mutation in the KCNQ1 gene, were
assessed.
This gene has been linked to the prevalence of atrial fibrillation in a Chinese
family afflicted with it.
The gene affects the \ii{Ks}\ channel, adding an instantaneous component to the
channel kinetics.
This additional leakage component dramatically abbreviates the action potential
duration. 
The restitution curves of the model were flattened and lowered by the influence of the
mutation, reducing the rate response of the tissue.
This allowed the tissue to sustain pacing at very high rates
($>$\unit{500}{bpm}) which were observed clinically.
Simulations in a two-dimensional tissue model revealed that the spatial
vulnerability window was reduced, suggesting that a much smaller region of
ectopic activity would be needed to incite arrhythmic activity.
The two dimensional models also suggested that the mutation allowed longer
lived spiral waves to exist in the tissue which would form a stable mother
rotor.
Simulations using the atrium model suggested that whilst in normal tissue
arrhythmic excitation would often decay away, in tissues afflicted with the
S140G mutation sustained arrhythmic activity was observed.
This activity was often a sustained mother rotor, but sometimes the anatomical
features present in the atrial model could lead to breakup into complex,
fibrilatory activity.

\section{Body Surface Potential Model}

A model of the P-wave body surface potential using the boundary element method
was created.
This used the atrial model as the basis for calculating the cardiac dipoles.
The meshes representing the body and selected internal inhomogeneities were
based on CT scans although they had been simplified considerably.
The atrium was located within the mesh using the internal inhomogeneities as
guides as well as anatomical descriptions of the location of the atria.
From the computed body surface potential traces were extracted to represent the
12 lead ECG and body surface potential maps.


Again, computational tractability was an important concern.
This motivated the choice to use the boundary element method.
In addition, the dipole contributions of several nodes of the atrial model were
aggregated together and assumed to act at the centroid of the aggregated region.
This reduced the total number of dipole sources by three orders of magnitude,
whilst having negligible impact on the final result.
The meshes were refined for the final computations to produce high quality body
surface potential maps.
This had almost no influence on the computed ECGs suggesting that for ECG
centric studies, such a refined mesh was not required.
Even with the refined mesh the problem remained tractable, solving \unit{1}{s}\
of body surface potential calculations in an acceptable time using only one
core.
The implementation itself is quite flexible, allowing for easy incorporation of
further meshes representing more internal detail.

Using the model the influence of internal inhomogeneities was examined.
It was found that the lungs tended to amplify the magnitudes of signals seen in
the leads, whilst the presence of blood masses tended to reduce them.
The inhomogeneities had little influence on the phase of the signals observed in
the 12 lead ECG, which showed good agreement with the clinical norms of the
P-wave ECG.
The amplitudes of the signal, in contrast, tended to be outside the normal
physiological range.
The reason for this was not entirely clear.
It could be due to the lack of consideration of certain inhomogeneities, such as
the skeletal bones or the skeletal muscle layer.
It might also be possible that a better model of the bloodmasses would reduce
this.

Despite the high amplitudes observed the model is still useful.
The phase relationships are generally of more interest clinically as are the
relative, rather than the absolute, amplitudes.
Because of the biophysically detailed basis of the atrial model, the body
surface potential model can be used in a variety of physiological studies to
provide insight into the clinically observable effects of the studied influence.

\section{Applications of the Body Surface Potential Model}

The body surface potential model was used to investigate two cases of clinical
interest.
It was used to assess an algorithm developed by physicians for locating the
origin of focal point tachycardia.
Also, a novel clinical phenomena, inverted P-waves at night, was investigated.

Focal atrial tachycardia is a supraventricular tachycardia characterised by
excitation emerging from an ectopic focus site.
It is readily treated by radio ablation therapy which is aided by accurate
knowledge of the focal site.
The algorithm presented used a decision tree approach, based on the waveform of
several 12 lead ECG leads.
To assess the algorithm, the model was paced from several ectopic sites around
the atria.
The ECG was used to make a decision based on the algorithm and this was compared
to the actual origin.
The origins predicted by the algorithm were in good agreement to the real pacing
locations in 6 out of 8 cases and showed only minor difference (left verses
right pulmonary veins) in one further case.
The final case, which is quite inaccurate, would probably be resolved were the
amplitudes closer to the clinically observed ranges--or if the algorithm's
threshold were adjusted to account for the higher potentials.
The good agreement provides validation of both the focal point origin algorithm
and the model itself.

Inverted P-waves at night are a phenomena which has not been reported on in the
literature.
They have been observed in the inferior limb leads in certain cases in patients
undergoing 24 hour cardiac monitoring.
It was hypothesised that this could be due to pacemaker shift in a distributed
pacemaker complex, induced by hormonal changes at night.
To investigate this hypothesis, the body surface potential model was paced from
several sites along the crista terminalis, to represent progressively greater
degree of shift and then the twelve lead ECG was computed.
It was observed that as the pacing site moved further along the crista
terminalis, the P-wave ECG in the inferior leads flattened and then inverted, in
accordance with the hypothesis.
Pacemaker shift is therefore a viable mechanism for the observed P-wave changes
at night.

\section{Future Work}

The work presented here opens up many paths for future efforts.
The models and tools developed in the study can be refined.
They can also be used, with and without such refinement, for new studies.

The toolkit developed can be extended in a number of ways.
The library of cellular models available can be increased, either through
direct addition, or by the development of a converter which can take CellML
models and produce code suitable for use.
More experimental protocols could be added to the toolkit, expanding its
utility.
The use of more sophisticated integrators for cellular equations could be
incorporated into the models and their influence on the performance and results
assessed.
The toolkit could also be expanded into three dimensions, allowing the same ease
of specifying numerical experiments to be enjoyed in whole heart simulations.
As an alternative to this, implementing the protocols on top of one of the of
the other toolkits might offer a better alternative.

The atrial model can also be refined in a number of ways.
As new models of atrial tissue are developed, they can be incorporated into the
model so that it always represents the best knowledge we have of the atrium and
its complexities.
A more complete map of the regions and directions of preferential conduction
could be incorporated into the model which could be important for non-sinus
rhythm cases.
For some studies an accurate and auto-active pacemaker complex would be a
benefit.
Contraction, and the associated mechanical and electric coupling, could be
incorporated into the model.

It might also be interesting to construct a limit variable formulation of the
model, either with Fenton--Karma variants or through variable reduction of the
CRN or other biophysically detailed models.
This would enable rapid initial investigation of interesting phenomena.
The full model would then be used to verify the initial findings and confirm
they still existed.

Beyond the atrium, the body surface model has several avenues for further study.
The influence of further inhomogeneities can be investigated.
Patient specific studies, where the cardiac geometry and torso structure are
based on CT and MRI scans of the patient offer exciting possibilities.
First, direct comparison with clinical data allows for further validation of the
underlying model.
Such models can also be used to suggest and test appropriate clinical
procedures.

Patient specific models also allow for the possibility of developing a family of models,
representing different orientations and conformations of the atrium and the
surrounding torso.
Using such a family of models would allow the results of computational studies
to be stated with more confidence.
An effect observed in one model might be an artefact of the geometry, but an
affect observed in five or ten or more is much less likely to be so.

Without any refinement, the toolkit and models developed offer a good basis for
many electrophysiological studies on a variety of scales.
The biophysical detail employed makes them suitable for the investigation of
complex effects.
The computational efficiency makes relatively long term simulations a
possibility too.
