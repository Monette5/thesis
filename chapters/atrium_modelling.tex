\chapter{Modelling the Atrium}

\section{Atrial Geometry}

The atrial geometry used in the simulation studies presented here was based on
the visible human project female dataset.  The visible human dataset was
created from a pair of cadavers, set into wax and sliced into \mm{1}\ and
\mm{0.33}\ for the male and female bodies, respectively.  The geometric model
used here was extracted from the female dataset and so has a resolution of
\mm{0.33}.  The extracted geometry is segmented into different tissue types,
with distinct classifications for left and right atrium, the pectinate muscles,
the crista terminalis, the Bachmann bundle and the sino-atrial node, as shown in
figure \ref{atrium:geometry}.  The geometry has been used in numerous previous
simulation studies.  It was discretised via a finite differences approach, which
allows the whole atrium to be embedded in a block of $298\times269\times235$
nodes.  This gives it a total size of approximately 18 million total nodes,
although only approximately 1.6 million of those nodes correspond to excitable
cells.  The geometry also has simple fibre orientation in the pectinate muscles,
crista terminalis and Bachmann bundle.  The fibres are considered to always run
parallel to the local axis of the tissue bundle, as determined by principle
component analysis~\cite{Seemann2006}.

\section{Simulation Methods}

\subsection{Atrial Model}

The electrical activity at each of the nodes was described by the equations of
the Courtemanche--Ramirez--Nattel (CRN) of the human atrial
myocyte~\cite{crn98}.  This model, as previously described, is a second
generation model. It has 21 state parameters, representing ionic gating activations
and inactivations and intracellular concentrations of ionic species.  In the
model, the total current, \ii{ion} is made up of the contributions of numerous
channels
\begin{equation}
\label{atrium:crn}
\ii{ion} = \ii{Na} + \ii{K1} + \ii{to} + \ii{Kur} + \ii{Ks} + \ii{Kr} +
\ii{Ca,L} + \ii{p,Ca} + \ii{NaK} + \ii{NaCa} + \ii{b,Na} + \ii{b,Ca}
\end{equation}
where \ii{Na}, \ii{K1}, \ii{to}, \ii{Kur}, \ii{Ks}, \ii{Kr}, \ii{Ca,L},
\ii{p,Ca}, \ii{b,Na} and \ii{b,Ca}\ represent ionic currents and \ii{NaK}\ and
\ii{NaCa}\ are ion exchangers.  As a second generation model, the CRN model also
has a detailed calcium handling system which can influence the action potential
via its influence on the intracellular calcium concentration.

In some atrial simulations it was desirable to incorporate details of
electrophysiological heterogeneity to represent the difference in electrical
behaviour between atrial myocytes and the other cellular types present in the
geometry, the pectinate muscles and crista terminalis.  The parameters used for
heterogeneity were based on measurements taken by Feng et al.~\cite{feng1998}
of the canine atrium.  These were converted to parameters for the CRN model by
Seemann et al.~\cite{Seemann2004} and have been used in several simulation
studies~\cite{Seemann2006,Stott2008}.  They are shown in
table~\ref{atrium:het_params}.

\subsection{Monodomain Equation}

To simulate the propagation of electrical activity over the finite difference
geometry previously described, the mono-domain equation is used to describe the
changes in $V$ in time, $t$, the trans-membrane voltage.
\begin{equation}
\label{atrium:monodomain}
\frac{\partial V}{\partial t} = \nabla\cdot D \nabla V - \frac{\ii{ion}}{C_{m}}
\end{equation}

where $D$ is a tensor representing the diffusivity, \ii{ion} is described by the
CRN model (\ref{atrium:crn}), $C_{m}$ is the membrane capacitance and all other
symbols have their usual meanings.  Equation (\ref{atrium:monodomain}) is
advanced in time via the forward euler method with a timestep of \ms{0.05}.  For
simulations with isotropic conductivity between nodes a 7-node approximation of
the differential operator is used.  When anisoptropy is present, a 27-node
approximation is used.

