\chapter{Modelling the Whole Atrium}

\section{Atrial Geometry}
\label{atrium:sec:geometry}

The atrial geometry used in the simulation studies presented here was based on
the visible human project female dataset.  The visible human dataset was
created from a pair of cadavers, set into wax and sliced into \mm{1}\ and
\mm{0.33}\ for the male and female bodies, respectively.  The geometric model
used here was extracted from the female dataset and so has a resolution of
\mm{0.33}.  The extracted geometry is segmented into different tissue types,
with distinct classifications for left and right atrium, the pectinate muscles,
the crista terminalis, the Bachmann bundle and the sino-atrial node, as shown in
figure \ref{atrium:geometry}.  The geometry has been used in numerous previous
simulation studies.  It was discretised via a finite differences approach, which
allows the whole atrium to be embedded in a block of $298\times269\times235$
nodes.  This gives it a total size of approximately 19 million total nodes,
although only approximately 1.6 million of those nodes correspond to excitable
cells.  The geometry also has simple fibre orientation in the pectinate muscles,
crista terminalis and Bachmann bundle.  The fibres are considered to always run
parallel to the local axis of the tissue bundle, as determined by principle
component analysis~\cite{Seemann2006}.

\section{Simulation Methods}

\subsection{Atrial Model}

The electrical activity at each of the nodes was described by the equations of
the Courtemanche--Ramirez--Nattel (CRN) of the human atrial
myocyte~\cite{crn98}.  This model, as previously described, is a second
generation model. It has 21 state parameters, representing ionic gating activations
and inactivations and intracellular concentrations of ionic species.  In the
model, the total current, \ii{ion} is made up of the contributions of numerous
channels
\begin{equation}
\label{atrium:crn}
\ii{ion} = \ii{Na} + \ii{K1} + \ii{to} + \ii{Kur} + \ii{Ks} + \ii{Kr} +
\ii{Ca,L} + \ii{p,Ca} + \ii{NaK} + \ii{NaCa} + \ii{b,Na} + \ii{b,Ca}
\end{equation}
where \ii{Na}, \ii{K1}, \ii{to}, \ii{Kur}, \ii{Ks}, \ii{Kr}, \ii{Ca,L},
\ii{p,Ca}, \ii{b,Na} and \ii{b,Ca}\ represent ionic currents and \ii{NaK}\ and
\ii{NaCa}\ are ion exchangers.  As a second generation model, the CRN model also
has a detailed calcium handling system which can influence the action potential
via its influence on the intracellular calcium concentration.

In some atrial simulations it was desirable to incorporate details of
electrophysiological heterogeneity to represent the difference in electrical
behaviour between atrial myocytes and the other cellular types present in the
geometry, the pectinate muscles and crista terminalis.  The parameters used for
heterogeneity were based on measurements taken by Feng et al.~\cite{feng1998}
of the canine atrium.  These were converted to parameters for the CRN model by
Seemann et al.~\cite{Seemann2004} and have been used in several simulation
studies~\cite{Seemann2006,Stott2008}.  They are shown in
table~\ref{atrium:het_params}.

\subsection{Monodomain Equation}

To simulate the propagation of electrical activity over the finite difference
geometry previously described, the mono-domain equation is used to describe the
changes in $V$ in time, $t$, the trans-membrane voltage.
\begin{equation}
\label{atrium:monodomain}
\frac{\partial V}{\partial t} = \nabla\cdot D \nabla V - \frac{\ii{ion}}{C_{m}}
\end{equation}
where $D$ is a tensor representing the diffusivity of electrical potential, \ii{ion} is described by the
CRN model (\ref{atrium:crn}), $C_{m}$ is the membrane capacitance and all other
symbols have their usual meanings.  Equation (\ref{atrium:monodomain}) is
advanced in time via the forward euler method with a timestep of \ms{0.05}.  For
simulations with isotropic conductivity between nodes a 7-node approximation of
the differential operator is used.  When anisoptropy is present, a 27-node
approximation is used.

\subsection{Tissue Anisotropy}

The heart has a complex fibrous structure (Chapter 1), and this manifests
electrically as regions which have preferential conduction directions.
The preferential conduction directions show greatly increased conduction
velocities, sometimes by a factor of up to five~\cite{}.
The fibre structure and regions of preferential conduction are generally
considered much more important for the ventricles than for the atria.
The atria, or more specifically the right atrium, do possess several structures
with a definite direction of preferential conduction.
These are the crista terminalis, responsible for rapid conduction of the
depolarization wave to the atrio-ventricular node, the pectinate muscles and the
Bachmann bundle, the preferential pathway for conduction between the atria.
To determine the influence of anisotropic conduction on the propagation of the
electrical activity, we follow a method after Panfilov and
Keener~\cite{panfilov1995}.
In this method there is a unit vector, $\mathbf{f}$, defined at every point in
the tissue which has significant fibre orientation.
This unit vector defines a set of co-ordinate axes, in which the conductivity
tensor is diagonal
\begin{equation}
\label{atrium:dtilde}
\mathbf{\tilde{D}} =
\begin{pmatrix}
D_{\parallel} & 0 & 0\\
0 & D_{\perp} & 0\\
0 & 0 & D_{\perp}
\end{pmatrix}
\end{equation}
where $D_{\parallel}$ is the diffusion constant for conduction parallel to the
preferential direction of conduction and $D_{\perp}$ is the diffusion constant
for conduction perpendicular to this direction.
In this formulation it is assumed that there is no `sheet' structure which gives
a higher conduction velocity in one direction perpendicular to the main fibre
axis.
The diffusion tensor $\mathbf{\tilde{D}}$\ will only be diagonal in the
Cartesian co-ordinate system of the heart if the direction of preferential
conduction is parallel to one of the axes.
Therefore, to find the conductivity tensor in the global co-ordinate system,
$\mathbf{D}$, we need to find two transformation matrices $\mathbf{A}$\ and
$\mathbf{A^{T}}$\ such that
\begin{equation}
\label{atrium:d}
\mathbf{D} = \mathbf{A} \mathbf{\tilde{D}} \mathbf{A^{T}}
\end{equation}
To find $\mathbf{A}$\ it is possible to write out the involved rotations
explicitly, however an alternative method~\cite{fention2005}\ uses the fact that
$\mathbf{f}$\ and the two vectors orthogonal to it, $\mathbf{g}$\ and
$\mathbf{h}$\ are eigenvectors of $\mathbf{D}$.
These have the eigenvalues of $D_{\parallel}$\ and $D_{\perp}$.
The matrix $\mathbf{A}$\ is therefore an orthogonal matrix of the form
$\mathbf{A} = \left(\mathbf{f},\mathbf{g},\mathbf{h}\right)$ and so, using
(\ref{atrium:d}) $\mathbf{D}$\ can be written as
\begin{equation}
\label{atrium:dfgh}
\mathbf{D} = D_{\parallel}\mathbf{f}\mathbf{f^{T}} +
D_{\perp}\left(\mathbf{g}\mathbf{g^{T}} + \mathbf{h}\mathbf{h^{T}}\right)
\end{equation}
Using the fact that $\mathbf{A}\mathbf{A^T} = \mathbf{I}$ it is possible to
write
\begin{equation}
\label{atrium:dwithf}
\mathbf{D} = D_{\perp}\mathbf{I} + \left(D_{\parallel}-D_{\perp}\right)\mathbf{f}\mathbf{f^{T}}
\end{equation}
where $\mathbf{I}$\ is the identity matrix, and all other symbols are as defined
previously.
The directions of preferential conduction for the atrial geometry used in the
study were described by a pair of angles $\theta$\ and $\phi$\ representing the
orientation of the unit vector $\mathbf{f}$\ at each point in spherical polar
co-ordinates.
In cells with no assigned preferential conduction direction, the components of
$\mathbf{f}$\ were set to zero, giving a diffusion tensor of
\begin{equation}
\label{atrium:dnofibre}
\mathbf{D} =
\begin{pmatrix}
D_{\perp} & 0 & 0\\
0 & D_{\perp} & 0\\
0 & 0 & D_{\perp}
\end{pmatrix}
\end{equation}
which is the diffusion tensor for isotropic conduction.

\subsection{Computational Implementation}

The atrial geometry used in these studies is quite large, consisting of almost
19 million nodes.
As noted in \ref{atrium:sec:geometry}, only approximately 1.6 million of these
nodes correspond to active tissue--less than 10\% of the total.
The electrical activity at each node is represented by the CRN model and thus
requires 21 double precision numbers to be stored, representing the state
variables of the model.
The memory requirements of the model may be significantly reduced by storing
state variables, and where anisotropy is present the diffusion tensor, only for
the active nodes.
This reduces the memory requirements for storing the state variables from
approximately \unit{2.9}{GB}\ to \unit{256}{MB}.
A further simplification may be obtained by decomposing the geometry into a
linear array, containing the 7 or 27 neighbours of the active nodes to be used
in the diffusion tensor approximation.
The geometry and state information can therefore be represented by one linear
array of cellular states, one linear array used as a `map' and optionally, one
linear array representing the components of $\mathbf{D}$.
This linear data structure is very easy to parallelize on a shared memory
system.

The parallelization was accomplished through the use of the OpenMP shared
memory parallelism library~\cite{OpenMP}.
The system was then solved on 1 node of the Horace supercomputer on a total of 8
cores.
The linear array of active nodes was divided equally between the 8 cores, with
each core solving (\ref{atrium:crn}) for all nodes its assigned section of the
array.
A snapshot of the trans-membrane potentials at each of the active node sites was
output every \ms{2.5}\ of simulated time.
Simulation of \unit{1}{s}\ of atrial activity took XXXX hours.
A parallel fraction of XXXX was attained, indicating that almost all of the
workload was effectively distributed over the 8 cores.

\section{The KCNQ1 Mutation: A Simulation Study}

Familial Atrial Fibrillation is blaaah.



