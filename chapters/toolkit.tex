\chapter{Constructing a Cardiac Simulation Toolkit}

\section{Simulation Environment}

The simulation environment provided by the cardiac toolkit is intended to be as
portable as possible, so that numerical experiments may be run on whichever
platforms are appropriate.  To this end, all the data input structures are based
on open standards, or simple binary formats.  The output formats provided by the
various driver programs are also in simple binary or ASCII formats, to allow
them to be easily visualized with both commercial and open source visualisation
tools.  The results presented later in this chapter were performed on desktop
computers with a XX GHz Althon X2 chip and 1 GB RAM and on Horace, the local HPC
facility.  Horace has 24 compute nodes, each one consisting of four Intel Itanium2
Montecito Dual Core 1.6GHz processors, 16GB RAM and up to 512GB of local scratch
space.  The nodes are connected by a high speed Quadrics QsNetII
interconnect~\cite{horace}.  Horace provides compilers for both Fortran and C,
and for both the MPI and OpenMP parallelization libraries.

\subsection{Implementation}

The experimental protocol drivers and the cellular models were implemented in
the C programming language, although much of the supporting code and
supplementary tools were implemented in the ruby programming language.  The
cellular models currently implemented are based on the Hodgkin-Huxley formalism,
although there is no fundamental reason why a Markov chain based model could not
be included.  Inter-cellular coupling for propagation of excitation over a
strand or tissue was implemented using the monodomain equations.

\subsubsection{Cellular Models}

The cellular model used for much of the developmental process was the
Courtemanche et al. human atrial myocyte model~\cite{crn98}.  Also currently
implemented are the Nygern et al. human atrial myocyte model~\cite{nygern98} and
the four variable formulation of the Fenton-Karma minimal variable
model~\cite{overo2008}.  These cellular models describe the behaviour of a cell
using coupled systems of non-linear ordinary differential equations.  The ODEs
represent the concentrations of intra- and extra-cellular ion species and the
flow of current through ionic channels in the cell membrane or between
intra-cellular compartments (Or their notional equivalents in the minimal
variable model)

These equations were generally solved using the simplest time-stepping method
available, the explicit Euler method.  To improve performance and stability,
some variables were integrated using the Rush-Larsen method.  More complex
integration schemes were tried, but did not significantly improve performance to
compensate for the greater complexity.

\subsubsection{Monodomain Equations}

The monodomain equations were used to couple multiple cells together to describe
a tissue over which excitation could be conducted.



The monodomain equations are considerably simpler than the bidomain equations
which have an elliptical component, representing the extra-cellular potentials,
that must be solved via iterative methods.  This additional complexity is not
required unless the desire is to model situations where there is significant
coupling to external voltage sources, for example to model defibrillation
techniques.

\subsection{Parallelization}

Some parts of the toolkit require the modelling of large numbers of cells, on
the order of tens or even hundreds of thousands of cellular models in two
dimensional sheets.  Solving all the equations involved takes a significant
amount of time and so it is desirable for such simulations to be parallelized so
that the work involved can be split over several processors.  This can have
advantages beyond merely having eight rather than one cores worth of
computational cycles working on solving the equations.  Splitting the work over
multiple cores can also increase the amount of cache available, allowing for
more efficient operation of the solvers.

For the toolkit, the OpenMP~\cite{OpenMP} parallelization library was chosen.
This implements the shared memory parallelism paradigm.  Using the shared memory
paradigm is simpler, due to the lack of explicit communication calls, and can be
faster, as there is merely memory access involved, rather than communication
over an network or other interconnect.  Finally, a program written using OpenMP
can, with some care, also be compiled serially if an OpenMP compiler is not
available.  Whilst a program implemented using a non-shared memory paradigm,
such as using the MPI library, would allow it to be executed on potentially more
processors which would potentially offer faster run times, in practice the eight
core shared memory system Horace provides as a compute node was general found to
be sufficient for the task.

\subsection{Optimisation}

The toolkit uses several techniques to optimize the computations performed,
reducing the number of processing cycles required to compute the results needed
for publication.  These optimisations are generally built around the principle
of ``No code is faster than no code'', although it might be more accurate to say
``No code execution is faster than no code executed''.  To explain this idea
another way, it is that no combination of compiler flags can compete with a
sensible choice of algorithm which minimizes the number of computational steps
to be performed or the amount of data to be written to a file.  This is not to
say that the compiler is are unimportant, but that it is merely the start of
optimisations, rather than the final step.

\subsubsection{The Compiler}

The toolkit has been compiled using the GNU C compiler (gcc)~\cite{gcc}, the
Intel C compiler (icc)~\cite{icc} and the Sun microsystems .  All three have
OpenMP implementations available and all three are capable of performing a
number of optimisations, controlled via flags.  The most important aspect of the
optimisations is that they should not alter the behaviour of the floating point
handling, as this could have significant impact on the final result computed.
Despite this caveat, the results of applying certain optimisation flags can be
quite significant.

\subsubsection{Caching of Computed Values}

Moving beyond the compiler, one of the simplest forms of optimisation is to only
calculate each value once, if at all possible.  This can be done in a number of
ways and the toolkit developed here implements two such methods for saving
computational time.

State saving is one of the most direct ways of caching computed values.  At a
particular point in the simulation, all of the state variables of the
system are copied into an intermediate location.  This might be a file on disk or
to another location in memory.  If the state is written out to a file, that file can
be used as a `save point', allowing the simulation to be continued from that
point in the future, ensuring work is not wasted.

When copied to another memory location, this allows the program to return to
that point in the future.  This is useful in modelling many experimental
protocols, which often call for a number of `conditioning' pulses to allow the
cell or model to settle.  The state can be saved after the conditioning pulses
and then the actual tests can be performed quickly, saving the execution of
several seconds of simulated activity.  This technique should obviously only be
used for cells in the Hodgkin-Huxley formalism which are deterministic and thus
give identical results whether the state is saved or not.  Using such a
technique with a cell that has a number of stochastic components could
potentially affect the quality of the results.

The second way in which caching can be employed is in the creation of `lookup
tables'.  A lookup table is a pre-computed table of the values an expression can
take.  When the expression would normally be evaluated, the table is used
instead, replacing what might be a complicated expression with a single array
lookup.  For lookup tables to be efficient to pre-compute, the tabulated
expression should depend on only one variable and should be sufficiently
`complex' expression, such as one involving the computation of mathematical logs
or exponentials.  The expression should depend on only one variable as the
pre-computed table has to be indexed over each variable the computation depends
on and even dependence on just two variables would increase the number of table
entries required by thousands or millions.  The requirement for complexity is a
little more obvious, as any optimisation can only save as much time as the
original series of operations took.  With these limitations in mind, it is still
possible to find many candidates for pre-computation in typical cardiac cell
models, most notably the activation and inactivations of voltage dependent
gates.  This technique is typically only worthwhile performing in the case of
tissue simulations where the benefits can be shared amongst all the cells since
pre-computation can be quite expensive, destroying any efficiency gains made
through their use for single cell simulations.

\subsubsection{Binary Searches}

Several of the experimental protocols provided by the toolkit are intended to
determine the value of a parameter which causes a particular condition to be
fulfilled, such as a successful excitation of the cellular model after
progressively shortening stimulus intervals.  This value we will call the
critical value. In real experiments, ones involving actual cardiac tissue, the
typical experimental protocol would involve stimulating the tissue at
sequentially shorter intervals, until no stimulation was provoked.  This might
involve stimulating the cell thousands of times, which would be expensive
computationally to model exactly.  Instead, a binary search for the critical
value can be performed, using the pseudo-code shown here.


To explain in words, first two guesses are made; the high guess, which is the
maximum value that the critical value can take, and the low guess, the minimum
it is presumed to take.  The simulation is then run with the parameter set at
the average of the low and high guesses--the current guess.  If the test is
successful, the critical value evidently lies somewhere between the low guess
and the average, and so the high guess is set to the current guess.  Conversely,
if the test is unsuccessful, the critical value is obviously above the current
guess, and so the low guess is set to the current guess.  The simulation is then
repeated with the average of the new high and low guess.  Using this algorithm,
the search space is halved with each iteration, swiftly finding the critical
value.  For example, to find a parameter somewhere in the range of 0--1000~ms to
the nearest millisecond requires just 10 iterations of the binary search
algorithm, but might take hundreds of iterations with sequential searching.

One important thing that must be considered when using binary searches is that
there is only one critical value in the range considered or else the algorithm
will give unpredictable results.  In practice, this limitation is often quite
easy to work within.

\subsubsection{Adaptive Step}

Adaptive step mechanisms are employed in the toolkit when there is a need to
provide output over a wide range of times, when the slope of the graph is not
constant over the range to be graphed.  This is very common in the modelling of
cardiac cells, which often show an exponential dependence of various parameters
on the  stimulus interval, and are graphed over a range of hundreds or thousands
of milliseconds.  A step that sufficient to track the curve at the upper limits
of the range will completely fail at the steeper slow of the lower limits,
whilst a step that will track the curve for the lower limits will result in
unnecessary work being done at the upper end of the range.  To alleviate this
problem, an adaptive stepping mechanism is used, as shown in this pseudo-code.

First, the measurement is performed at the largest desired point.  The interval
is then reduced by the step, and the measurement is performed again.  The
difference in the measurements is calculated and compared to the desired maximum
delta.  If the difference is acceptable, the interval is once more reduced by
the step, and the measurement taken once more.  If the difference is too great,
then instead the step size is halved and the measurement repeated.  If the
difference is now acceptable, then the interval is reduced by the new step and
the experiment proceeds.  If it is not, then the step size is once more halved.
The step size used is therefore always appropriate to the slope of the curve and
a smooth graph results.  Additional logic, not shown in the pseudo-code, is used
to ensure the step size does not become too small, and to terminate the graph at
the lower end of the range.

Since curves can increase or decrease the absolute difference between the two
values is compared.

\subsubsection{Parallel Output}


\section{Experimental Protocols}



\subsection{Action Potential Duration}

