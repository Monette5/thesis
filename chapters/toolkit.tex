\chapter{Constructing a Cardiac Simulation Toolkit}

\section{Simulation Environment}

The simulation environment provided by the cardiac toolkit is intended to be as
portable as possible, so that numerical experiments may be run on whichever
platforms are appropriate.  To this end, all the data input structures are based
on open standards, or simple binary formats.  The output formats provided by the
various driver programs are also in simple binary or ASCII formats, to allow
them to be easily visualized with both commercial and open source visualisation
tools.  The results presented later in this chapter were performed on desktop
computers with a XX GHz Althon X2 chip and 1 GB RAM and on Horace, the local HPC
facility.  Horace has 24 compute nodes, each one consisting of four Intel Itanium2
Montecito Dual Core 1.6GHz processors, 16GB RAM and up to 512GB of local scratch
space.  The nodes are connected by a high speed Quadrics QsNetII
interconnect~\cite{horace}.  Horace provides compilers for both Fortran and C,
and for both the MPI and OpenMP parallelization libraries.

\subsection{Implementation}

The experimental protocol drivers and the cellular models were implemented in
the C programming language, although much of the supporting code and
supplementary tools were implemented in the ruby programming language.

\subsection{Optimisation}

\subsection{Parallelization}



\section{Experimental Protocols}



\subsection{Action Potential Duration}

