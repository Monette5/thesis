\section{Cardiac Simulation Toolkit}

To extract useful results from cardiac modelling two principle components are
needed, as has been previously mentioned: the mathematical models used for
cardiac activity and a computational representation of the experimental
protocols followed to produce results.  A cardiac simulation toolkit provides
one or both of these components to an investigator, removing the need for the
investigator to implement those components themselves.  The word toolkit is
chosen to differentiate from the relatively more common practice of just
providing individual cell models.  A toolkit implies more than one model, or
variants on a model, with a common calling interface and implementation of one
or more experimental protocols or a non-programmatic way of setting up such
protocols---whether through a GUI or through command-line options or
configuration files.

Toolkits have a number of advantages to the investigator, including a reduction
in the work required to bring a paper to publication, increased consistency both
within and outside of the group, fewer implementation mistakes, increased
possibilities for collaboration and verification and a greater accessibility to
cardiac simulations for non-programmers.  The disadvantages of toolkits are not
as numerous, but they too will be considered.  A toolkit will need auditing, to
ensure that the results it gives are accurate.  Toolkits can also constrain the
avenues of experimentation pursued by an investigator by making protocols easier
or harder to implement.  Finally, the additional complexity makes adding any new
feature a little harder---there is a `feature cost'.

\subsection{Advantages of Cardiac Simulation Toolkits}



Cardiac simulation toolkits have existed in one form or another for some
time and include both commercial and open source 
Several cardiac simulation systems have previously been released, one of
the first of which is the OXSOFT HEART~\cite{Noble-1999}.
