\section{Cardiac Simulation Toolkit}

To extract useful results from cardiac modelling two principle
components are needed, as has been previously mentioned: the
mathematical models used for cardiac activity and a computational
representation of the experimental protocols followed to produce
results.  A cardiac simulation toolkit provides one or both of these
components to an investigator, removing the need for the investigator to
implement those components themselves.

The
advantages to the investigator vary from toolkit to toolkit, but the
major one is

Cardiac simulation toolkits have existed in one form or another for some
time and include both commercial and open source 
Several cardiac simulation systems have previously been released, one of
the first of which is the OXSOFT HEART~\cite{Noble-1999}.
