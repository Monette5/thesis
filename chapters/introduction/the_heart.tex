\section{The Heart}

The heart's role is to pump blood around the body, driving in the circulation of
the blood and everything contained within it.  It is one of the most important
organs in the body and any malfunction in its behaviour could be fatal in very
short order.  It begins beating in the early stages of pregnancy and continues
until death, hopefully many decades later.  It beats at an average rate of
around 70 beats per minute (bpm) for the adult male and 75 bpm for the adult
female.

The heart is not, as popular belief would have it, the seat of human emotion.
The functioning of the heart is modulated by such emotion however, slowing when
we are calm and increasing in rate quite dramatically when we are excited or
afraid.  Despite being influenced by the brain and our emotional states, the
heart drives itself, rather than having the pace-making initiated outside the
organ.

\subsection{Location of the Heart}

The human heart sits in the centre of the chest, the bulk of it extending the
left-hand side of the chest cavity, inside a fibrous sac called the pericardium.

\subsection{Structure of the Heart}

The structure is mostly muscle, anchored to a collagenous `skeleton', known as
the annulus fibrosus located at the atrio-ventricular junction.  This muscle is
different from the `smooth' skeletal muscle, in both structure and behaviour.

\subsubsection{The Four Chambers}

The heart has four chambers, two atria and two ventricles.  The atria receive
the blood from the circulatory system and force it into the two ventricles,
which then contract and force this blood out and around the lungs and body.
These chambers are known as the left and right atria and the left and right
ventricles.  The left hand side of the heart in humans is much more developed
than the right.  This is due to their differing roles in the circulation of the
blood.

The right and left atria are smaller than their respective ventricles and have
much thinner walls, because they need to develop much less pressure.  The right
atrium receives the blood from the circulatory system which is de-oxygenated,
and passes it on to the right ventricle.  The left atrium receives the highly
oxygenated blood from the lungs, and passes it onto the left ventricle.  The two
atria are separated by a thin muscle wall known as the intra-atrial septum.
This prevents the mixing of blood between the two atrial chambers.

The differences between the right and left ventricles are much more pronounced
than those between the right and left atria.  The right ventricle must merely
pump blood around the lungs and developing too high a pressure there could
actually damage the delicate structures.  By contrast the left ventricle must
develop enough pressure to drive blood around the whole body and as such it is
much more muscled.  Again, the two ventricles are divided by the ventricular
septum.

\subsubsection{The Fibrous Structure}


