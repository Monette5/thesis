\section{Activation of the Heart}

In normal function the heart depolarises in a rhythmic and controlled fashion,
at a rate of 60 to 100 bpm.
There exist numerous abnormal rhythms, or arrhythmias, which are diagnosed in
patients with heart disease.
These can be divided into two broad categories.
Slowed heart rate, or bradycardia, with a heart rate below 60 bpm and quickened
heart rate, tachycardia, with a sustained heart rate over 100 bpm.
Bradycardia and tachycardia manifest in differing ways in the atria and
ventricles; this section focuses on atrial manifestations.

\subsection{Normal Activation of the Heart}

The normal heartbeat originates in the right atrium, in the sino-atrial node.
This region of auto-active cells regularly depolarises, sending out excitation
waves through the cardiac tissue.
These excitation waves spread in all directions along the atrial walls, though
they are preferentially conducted along the crista terminalis to another area of
specialised cells, the atrio-ventricular node.
The atrio-ventricular node slows conduction through it, due to small and low
capacitance cells.

While the atrio-ventricular node is depolarising, the atria finish their own
depolarisation.
In the right atrium, this spreads from the sino-atrial node and the fast
conducting muscle ridges of the crista terminalis and pectinate muscles.
Excitation is conducted to the left atrium through the Bachmann bundle or
an inferior muscle bundle.
The atrial depolarisation leads to the atria contracting and charging the
ventricles with blood.

The atrio-ventricular node depolarising carries the excitation through the
central fibrous body and into the bundle of His.
The bundle of His is made of fast conducting muscle fibres and soon splits into
the left and right bundle branches, which form the start of the purkinje fibre
network.
The purkinje fibres are made of another specialised cell type, and they conduct
the electrical signal quickly through the body of the ventricular muscle.
The purkinje fibres break out on the endocardial surface of the ventricles, the
only place they are electrically coupled to the normal ventricular muscle.

As the excitation wave emerges from the breakout point, it rapidly conducts
through the thickness of the ventricular wall to the epicardial surface.
This creates a powerful contraction which forces the blood from the ventricles.
The ventricles then depolarise, as a result of the complex action potential
heterogeneity in the ventricular walls, in much the same direction; from the
inside out.

\subsection{Mechanisms of Bradycardia}

In the atrium, bradycardia, slow heart rate, is most commonly due to increased
parasympathetic (or vagal) tone.
This can actually be `normal' in the case of athletes, whose training can result
in them having very low resting heart rate.
Other causes can include cardiac drugs, such as beta blockers, or hypothyroid.

\subsubsection{Sick Sinus Syndrome}

Sick sinus syndrome is a grouping of conditions, most common in the elderly.
In conditions of sick sinus syndrome, the bradycardia is pronounced enough to be
dangerous.
As a grouping of conditions, sick sinus syndrome has a number of causes.
Congenital sick sinus syndrome has been linked to mutations of the SCN5A
protein~\cite{Benson2003}, which forms part of the sodium channels in the sinus
node.
In elderly patients, the cause can be more varied.
The condition has been linked to increased fibrosis of cardiac
tissue~\cite{Kohl2005}\ and also due to age related changes in sinus node
electrophsyiology~\cite{Alings1993}.

\subsection{Mechanisms of Tachycardia}

\subsubsection{Re-entry}

\subsubsection{Spiral Waves}

\subsubsection{Ectopic Foci}
