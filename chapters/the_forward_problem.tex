\chapter{Applications Of The Forward Problem}

The ECG is the first tool cardiac doctors turn to when diagnosis of a problem is
required.
ECG machines can be found in almost every hospital in the world.
A model of the atria and the surface potentials developed by the excitation of
the model can be used to guide diagnosis of a variety of conditions.
This can reduce the need for surgical procedures or suggest when they are
essential.

Whilst the inverse problem promises to reproduce the potentials on the heart
from the potentials on the surface, the technique has limitations.
For accurate solutions, patient specific geometries have to be constructed from
MRI scans.
There is also a need for complex lead systems, sometimes featuring more than two
hundred leads.
Also, many of the inverse techniques rely on `smooth' propagation patterns to
reduce the uncertainties in the technique which may not be found in pathological
cases.
A device which can perform such calculations automatically is a long way off,
both in terms of computational power required and complexities to resolve.

By contrast, diagnostic guides based on a forward solution can be of use to any
doctor.
They can also be used to further validate simulation studies of genetic or
diseased conditions, by comparison of the generated ECGs with those recorded
from real patients.
This chapter explores some of these predictions, using the model developed in
the previous chapter.

\section{Inverted P-Waves at Night}

Recently an observation was made~\cite{BoyettPrivate}\ concerning patients under
24 hour ECG monitoring.
It was noted that some patients exhibited inverted P-waves at night.
That is to say, if the patient showed a positive P-wave in leads II and aVF
during the day, then at night the P-wave would be negative in leads II and aVF.
This phenomena has not been reported in the literature.

There is evidence~\cite{Shibata2001,Boineau1988} that the pacemaker is not a
small and discrete area of the atrium, but is instead distributed along the
length of the crista terminalis.
The presence of certain drugs and hormones, most notably acetylcholine, can cause
the site of the leading pacemaker to move down the pace maker complex.
Acetylcholine is released by what is know as increased `vagal tone'.
This has been observed to happen at night.
It was hypothesised that a pacemaker shift induced by this increased vagal tone
might lead to the observed P-wave inversion.

\subsection{Methods}

In the absence of a model for the distributed pacemaker complex in the human
heart, the direct effects of acetylcholine could not be investigated.
Instead, using the model presented in the previous chapter, several sites were
located along the crista terminalis.
These sites had a radius of 15 nodes (or approximately \mm{5}--although this
varied depending on the thickness of the atrial wall at the pacing site), and
therefore were approximately the same size as the sino-atrial node.
These sites are shown in figure~\ref{fig:forward:ct_sites}.
Each of these sites was stimulated via the same protocol used to stimulate the
sinus node in the original model and then the electrical excitation waves were
allowed to propagate without interference.

ECGs were computed from the patterns of electrical excitation in the atrium.
These were compared to the sinus rhythm ECGs computed in the previous chapter.
In addition, using a so called `inverse Dower' method after Edenbrandt and
Pahlm~\cite{Edenbrandt1988}, the orthogonal components of the ECG were computed
and used to construct representations of the heart
vector~\cite{Frank1956,MacFarlane1989a} ECG (VECG).
To perform the inverse dower transformation, a matrix that has been optimized for
the P-wave (shown in Table~\ref{tbl:forward:idparams}) was
used~\cite{Guillem2007}.


\begin{table}
\caption[Inverse Dower Factors]{
\label{tbl:forward:idparams}
Factors to construct the Frank VECG from the standard 12 lead ECG set.
Parameters optimised to accurately reproduce the P-wave heart
vector~\cite{Guillem2007}.
Each of the 8 leads are multiplied by the given parameters to provide the
orthogonal Frank lead.
}
\begin{center}
\begin{tabular}{c c c c c c c c c}
\toprule
& $\text{V}_{\text{1}}$ &$\text{V}_{\text{2}}$ & $\text{V}_{\text{3}}$ &
$\text{V}_{\text{4}}$ & $\text{V}_{\text{5}}$ & $\text{V}_{\text{6}}$ & I & II \\
\midrule
X & $-0.266$ & $\:0.027$ &  $\:0.065$ & $\:0.131$ & $\:0.203$ & $\:0.220$ & $\:0.370$ & $-0.154$ \\
Y & $\:0.088$ &  $-0.088$ & $\:0.003$ & $\:0.042$ & $\:0.047$ & $\:0.067$ & $-0.131$ & $\:0.717$ \\
Z & $-0.319$ & $-0.198$ & $-0.167$ & $-0.099$ & $\:0.009$ & $\:0.060$ & $\:0.184$ & $\:0.114$ \\
\bottomrule
\end{tabular}
\end{center}
\end{table}

\subsection{Results}

\subsubsection{Activation Sequence}

The activation sequences of the atria after pacing from the sinus node, and the
three sites along the crista terminalis are shown in
figure~\ref{fig:forward:inverse:active}\ as isochronal colour maps.
Time goes from red, at \ms{0}, to blue, at \ms{150}.
The site of first activation obviously shifts depending on the stimulus
location.
In addition, as the stimulus site moves away from the sinus node, the time to
total activation of the atria increases.

The sinus node activation sequence,
figure~\ref{fig:forward:inverse:active}(s)(i, ii), starts high on the right
atrium, close to the superior vena cava.
Conduction is especially rapid down the crista terminalis and along the
pectinate muscles, visible as the more widely spaced isochrones along these
structures.
The Bachmann bundle, meanwhile, conducts the electrical excitation to the left
atrium, where it then starts to spread over the left atrial endocardial surface.
In the right atrium the activation finishes, after approximately \ms{80}\ have
elapsed since stimulation started, with the activation of the ring around the
tricuspid valve and the right atrial appendage.
In the left atrium, the far extremities of the left atrial appendage and the far
side of the mitral valve to the Bachmann bundle are activated at approximately
\ms{120}, completing the activation of the atria.

The activation sequence from site A,
figure~\ref{fig:forward:inverse:active}(a)(i, ii), starts high on the right
atrium in the region where the pectinate muscles are branching from the crista
terminalis.
This leads to rapid conduction down the pectinate muscles and in both directions
along the crista terminalis.
The Bachmann bundle conducts the excitation to the left atrium, where it starts
to spread.
The activation of the right atrium finishes with the activation of the right
atrial appendage and then region between the septum and the tricuspid valve.
Activation of the left atrium completes in \ms{132}\ after the initial stimulus
on the edge of the left atrial appendage and the mitral valve.

Pacing from site B leads to the activation sequence depicted in
figure~\ref{fig:forward:inverse:active}(b)(i, ii).
The activation sequence starts lower on the crista terminalis, and is conducted
in both directions along the muscle ridge.
The pectinate muscles also influence the conduction, although the excitation
wavefront reaches them later.
Excitation still reaches the left atrium through the Bachmann bundle, although
it is also conducted through the septum close to the inferior vena cava.
The last point to be excited in the right atrium is still the appendage.
In the left atrium, the last activation comes at \ms{140}.
The left atrial appendage and the sheaths of the pulmonary veins both finish
activating at this time.

The activation sequence which results from pacing from site C is shown in
figure~\ref{fig:forward:inverse:active}(c)(i, ii).
The activation sequence starts low on the right atrium and spreads in all
directions, but is faster travelling up the crista terminalis.
The pectinate muscles, when the excitation wave reaches them, also have a
noticeable effect, speeding activation of the right atrium.
Once again, the left atrium appears to be excited in two places, both by the
bachmann's bundle and close to the inferior vena cava through the septum.
The right atrial appendage is the last region of the right atrium to be excited.
In the left atrium, the extremities of the appendage and the pulmonary vein
sheaths are the last to be excited, as well as the region close to the mitral
valve.
This activation starts at approximately \ms{140}.

All of the activation sequences are approximately normal, in that they travel
from right to left.
The further the stimulus site is removed from the sinus node, the longer
excitation tends to take.


\subsubsection{Twelve Lead ECG}

The ECGs from the three pacing locations along the CT are shown in
figure~\ref{fig:forward:inverse:ecgs}\ with the sinus rhythm P-wave ECG for
comparison.
A summary of the lead deflections for the four cases and sinus rhythm are
presented in table~\ref{tbl:forward:inverse:ecgs}.
There is a clear evolution of the P-wave morphology visible as the pacing site
is moved down the crista terminalis.

\begin{table}
\caption[Lead classification under pacing from different locations]{
\label{tbl:forward:inverse:ecgs}
Lead classification after pacing from different locations.
Leads are classified based on the criteria used by
Kistler~et~al.~\cite{Kistler2006}.
A positive P-wave is denoted by a $+$ sign, a negative P-wave by a $-$ sign and
a biphasic one by $\sim$.
Site denotes the pacing site, where S is the sinus node and Sites A--C are as
indicated in Figure~\ref{fig:forward:ct_sites}.
}
\begin{center}
\begin{tabular}{c c c c c c c c c c c c c}
\toprule
Site & I & II & III & aVR & aVL & aVF & $\text{V}_{\text{1}}$ &$\text{V}_{\text{2}}$ & $\text{V}_{\text{3}}$ & $\text{V}_{\text{4}}$ & $\text{V}_{\text{5}}$ & $\text{V}_{\text{6}}$\\
\midrule
S   & $+$ & $+$ & $\sim$ & $-$ & $+$ & $+$ & $-$ & $+$ & $+$ & $+$ & $+$ & $+$ \\
A   & $+$ & $+$ & $\sim$ & $-$ & $+$ & $\sim$ & $-$ & $+$ & $+$ & $+$ & $+$ & $+$ \\
B   & $+$ & $\sim$ & $-$ & $-$ & $+$ & $-$ & $+$ & $+$ & $+$ & $+$ & $+$ & $+$ \\
C   & $+$ & $-$ & $-$ & $-$ & $+$ & $-$ & $+$ & $+$ & $+$ & $+$ & $+$ & $+$ \\
\bottomrule
\end{tabular}
\end{center}
\end{table}

The P-wave ECG for pacing from the sinus node,
figure~\ref{fig:forward:inverse:ecgs}(s), is positive in leads II and aVF.
Leads I, aVL and $\text{V}_{\text{2--6}}$ are also positive.
Leads aVR and $\text{V}_{\text{1}}$ are negative.
Lead III is biphasic.
The largest positive lead is $\text{V}_{\text{3}}$, which attains a potential
difference of \mv{+0.390} and the largest negative lead is aVR,
which attains a potential difference of \mv{-0.299}.
The definite positive P-waves in the last 5 precordial leads,
$\text{V}_{\text{2--6}}$, suggest normal propagation through the atrium, with
depolarisation travelling from right to left.
The cardiac axis is approximately $+30^\circ$ in the frontal plane.


Pacing from site A (figure~\ref{fig:forward:inverse:ecgs}(a)), the P-wave ECG is
positive in leads II and  aVF.
Leads I, aVL and $\text{V}_{\text{2--6}}$ are also positive.
Lead aVR and $\text{V}_{\text{1}}$ are negative.
Lead III is biphasic.
The largest positive lead is $\text{V}_{\text{3}}$, which attains a potential
difference of \mv{+0.358} and the largest negative lead is aVR,
which attains a potential difference of \mv{-0.240}.
The definite positive P-waves in the last 5 precordial leads,
$\text{V}_{\text{2--6}}$, suggest normal propagation through the atrium, with
depolarisation travelling from right to left.
The cardiac axis is approximately $+30^\circ$ in the frontal plane.
There is a very prominent notch visible in both leads II and aVF, caused by ...
foo!

Pacing from site B (figure~\ref{fig:forward:inverse:ecgs}(b)), the P-wave ECG is
biphasic in lead II and negative in lead aVF.
Leads I, aVL and $\text{V}_{\text{1--6}}$ are positive.
Lead III and aVR are negative.
The largest positive lead is $\text{V}_{\text{2}}$, which attains a potential
difference of \mv{+0.356} and the largest negative lead is III,
which attains a potential difference of \mv{-0.258}.
The definite positive P-waves in the last 5 precordial leads,
$\text{V}_{\text{2--6}}$, suggest normal propagation through the atrium, with
depolarisation travelling from right to left.
The propagation up the crista terminalis, not down it, is evidenced in the
positive P-wave in $\text{V}_{\text{1}}$.
The cardiac axis is approximately $-30^\circ$ in the frontal plane, due to
activation now travelling up the atrium, rather than down.


Pacing from site C (figure~\ref{fig:forward:inverse:ecgs}(c)), the P-wave ECG is
negative in lead II and lead aVF.
Leads I, aVL and $\text{V}_{\text{1--6}}$ are positive.
In addition to leads II and aVF, lead III and lead aVR are negative.
The largest positive lead is $\text{V}_{\text{2}}$, which attains a potential
difference of \mv{+0.343} and the largest negative lead is III,
which attains a potential difference of \mv{-0.263}.
The positive P-waves in leads $\text{V}_{\text{2--6}}$ suggest that the
propagation is still from right to left, although the low amplitudes in the
later leads suggests that this is not as uniform as in the previous cases.
The propagation up the crista terminalis, not down it, is evidenced in the
positive P-wave in $\text{V}_{\text{1}}$.
The cardiac axis is approximately $-60^\circ$ in the frontal plane, due to
activation now travelling up the atrium, rather than down.

As the pacing site moves down the crista terminalis there is an evolution of the
P-wave, which is visible in all leads.
As a result of the shift, the cardiac axis shifts through about $90^\circ$
anti-clockwise from the sinus direction of $+30^\circ$.
The P-wave duration increases slightly as the crista terminalis increases,
highlighting the importance of the specialized conduction structures of the
heart in rapidly conducting the excitation wave.


\subsubsection{Derived Vector ECGs}

The derived vector ECG plots are shown for the frontal plane in
figure~\ref{fig:forward:inverse:vec_front} and in the transverse plane in
figure~\ref{fig:forward:inverse:vec_trans}.
Again, the sinus rhythm is included in both figures for reference.
The colour of the vector loop represents the passage of time and is coloured
from purple, at \ms{0}, through blue, green, yellow and ending up red at
\ms{500} after initiation of the P-wave.

\subsection{Limitations}

Edenbrandt and Pahlm verses Uijen, optimised sets.  Perhaps Hyttinen paper.
Optimised for ST segment, not P-wave.
The original inverse dower matrix, as presented by Edenbrandt and
Pahlm~\cite{Edenbrandt1988}\ was used to perform the transformations, as used in
previous studies~\cite{Carlson2005,Holmqvist2007,Havmoller2007}

\section{Focal Atrial Tachycardia}

Atrial Tachycardias are one of the rarer forms of supraventricular
tachycardia, accounting 
They tend to occur as a result of other cardiac or respiratory diseases.
They are characterised by a high heart rate ($\geq$ \unit{250}{bpm}) and
typically have evidence of an abnormal cardiac axis or P-wave morphology.
They are hard to treat with drugs, but radiofrequency ablation can be used with
a high probability of success.
Diagnosis of atrial tachycardia can be difficult due to a lack of data.
Attempts to locate the sites of the ectopic focus are current topics of clinical
research~\cite{Kistler2006,Kahn2006,Yamane2001}.
This study shows how the model can provide more data for clinicians to study.

\subsection{Model of Focal Atrial Tachycardia}

To model focal atrial tachycardia, the model developed in the previous chapter
is used.
Instead of pacing from the cells corresponding to the sinus node, various sites
around the atria are selected.
These sites are shown in Figure~\ref{fig:forward:sites}.
All the nodes which have active cells within 10 cells (\mm{3.3}) are excited via
direct current injection of \unit{2}{nS}\ for \ms{2}.
The resulting excitation wave is then allowed to propagate without interference
and the BSPM and ECG calculated.
The ECGs are classified by the P-wave morphology and the time of first
deflection in each lead.



