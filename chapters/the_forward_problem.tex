\chapter{Applications Of The Forward Problem}

The ECG is the first tool cardiac doctors turn to when diagnosis of a problem is
required.
ECG machines can be found in almost every hospital in the world.
A model of the atria and the surface potentials developed by the excitation of
the model can be used to guide diagnosis of a variety of conditions.
This can reduce the need for surgical procedures or suggest when they are
essential.

Whilst the inverse problem promises to reproduce the potentials on the heart
from the potentials on the surface, the technique has limitations.
For accurate solutions, patient specific geometries have to be constructed from
MRI scans.
There is also a need for complex lead systems, sometimes featuring more than two
hundred leads.
Also, many of the inverse techniques rely on `smooth' propagation patterns to
reduce the uncertainties in the technique which may not be found in pathological
cases.
A device which can perform such calculations automatically is a long way off,
both in terms of computational power required and complexities to resolve.

By contrast, diagnostic guides based on a forward solution can be of use to any
doctor.
They can also be used to further validate simulation studies of genetic or
diseased conditions, by comparison of the generated ECGs with those recorded
from real patients.
This chapter explores some of these predictions, using the model developed in
the previous chapter.

\section{Inverted P-Waves at Night}

Recently an observation was made~\cite{BoyettPrivate}\ in patients under 24 hour
ECG monitoring.
It was noted that some patients exhibited inverted P-waves at night.
That is to say, if the patient showed a positive P-wave in leads II and aVF
during the day, then at night the P-wave would be negative in leads II and aVF.
This phenomena has not been reported in the literature.

There is evidence~\cite{Shibata2001,Boineau1988} that the pacemaker is not a
small and discrete area of the atrium, but is instead distributed along the
length of the crista terminalis.
The presence of certain drugs and hormones, most notably astylcholin, can cause
the site of the leading pacemaker to move down the pace maker complex.
Astylcholin is released by what is know as increased `vagal tone'.
This has been observed to happen at night.
It was hypothesised that a pacemaker shift induced by this increased vagal tone
might lead to the observed P-wave inversion.

\subsection{Methods}

In the absence of a model for the distributed pacemaker complex in the human
heart, the direct effects of astylcholin could not be investigated.
Instead, using the model presented in the previous chapter, several sites were
located along the crista terminalis.
These sites had a radius of 15 nodes (or approximately \mm{5}--although this
varied depending on the thickness of the atrial wall at the pacing site), and
therefore were approximately the same size as the sino-atrial node.
These sites are shown in Figure~\ref{fig:forward:ct_sites}.
Each of these sites was stimulated via the same protocol used to stimulate the
sinus node in the original model and then the electrical excitation waves were
allowed to propagate without interference.

ECGs were computed from the patterns of electrical excitation in the atrium.
These were compared to the sinus rhythm ECGs computed in the previous chapter.
In addition, using a so called `inverse Dower' method after Edenbrandt and
Pahlm~\cite{Edenbrandt1988}, the orthogonal components of the ECG were computed
and used to construct representations of the heart
vector~\cite{Frank1956,MacFarlane1989a} ECG (VECG).
To perform the inverse dower transformation, a matrix that has been optimized for
the P-wave (shown in Table~\ref{tbl:forward:idparams}) was
used~\cite{Guillem2007}.


\begin{table}
\caption[Inverse Dower Factors]{
\label{tbl:forward:idparams}
Factors to construct the Frank VECG from the standard 12 lead ECG set.
Parameters optimised to accurately reproduce the P-wave heart
vector~\cite{Guillem2007}.
Each of the 8 leads are multiplied by the given parameters to provide the
orthogonal Frank lead.
}
\begin{center}
\begin{tabular}{c c c c c c c c c}
\toprule
& $\text{V}_{\text{1}}$ &$\text{V}_{\text{2}}$ & $\text{V}_{\text{3}}$ &
$\text{V}_{\text{4}}$ & $\text{V}_{\text{5}}$ & $\text{V}_{\text{6}}$ & I & II \\
\midrule
X & $-0.266$ & $\:0.027$ &  $\:0.065$ & $\:0.131$ & $\:0.203$ & $\:0.220$ & $\:0.370$ & $-0.154$ \\
Y & $\:0.088$ &  $-0.088$ & $\:0.003$ & $\:0.042$ & $\:0.047$ & $\:0.067$ & $-0.131$ & $\:0.717$ \\
Z & $-0.319$ & $-0.198$ & $-0.167$ & $-0.099$ & $\:0.009$ & $\:0.060$ & $\:0.184$ & $\:0.114$ \\
\bottomrule
\end{tabular}
\end{center}
\end{table}

\subsection{Results}

The ECGs from the three pacing locations along the CT are shown in
Figure~\ref{fig:forward:inverse:ecgs} and summary of the lead deflections are
presented in Table~\ref{tbl:forward:inverse:ecgs}, along with the ECG generated
from pacing at the sinus node for comparison.
It is immediately obvious that even the small movement to site A for pacing
causes an inversion in the P-wave in lead aVF, from positive to negative.
Whilst pacing from this site, the P-wave in lead II is still not negative and is
instead biphasic.

\begin{table}
\caption[Lead classification under pacing from different locations]{
\label{tbl:forward:inverse:ecgs}
Lead classification after pacing from different locations.
Leads are classified based on the criteria used by
Kistler~et~al.~\cite{Kistler2006}.
A positive P-wave is denoted by a $+$ sign, a negative P-wave by a $-$ sign and
a biphasic one by $\sim$.
Site denotes the pacing site, where SAN is the sinus node and Sites A--C are as
indicated in Figure~\ref{fig:forward:ct_sites}.
}
\begin{center}
\begin{tabular}{c c c c c c c c c c c c c}
\toprule
Site & I & II & III & aVR & aVL & aVF & $\text{V}_{\text{1}}$ &$\text{V}_{\text{2}}$ & $\text{V}_{\text{3}}$ & $\text{V}_{\text{4}}$ & $\text{V}_{\text{5}}$ & $\text{V}_{\text{6}}$\\
\midrule
SAN & $+$ & $+$ & $\sim$ & $-$ & $+$ & $+$ & $-$ & $-$ & $+$ & $+$ & $+$ & $+$ \\
A   & $+$ & $\sim$ & $-$ & $-$ & $+$ & $-$ & $\sim$ & $+$ & $+$ & $+$ & $+$ & $+$ \\
B   & $+$ & $\sim$ & $-$ & $-$ & $+$ & $-$ & $\sim$ & $+$ & $+$ & $+$ & $+$ & $+$ \\
C   & $+$ & $-$ & $-$ & $\sim$ & $+$ & $-$ & $\sim$ & $\sim$ & $+$ & $+$ & $+$ & $+$ \\
\bottomrule
\end{tabular}
\end{center}
\end{table}

The sinus node ECG has already been described in some detail (Section XXX), and
so only a brief summary will be presented here.
The P-wave is upright in leads II and lead aVF.
The first precordial lead, $\text{V}_{\text{1}}$ is predominately negative, as
the electrical excitation is moving away from the electrode as the
depolarisation wave travels downwards from the sinus node at the top.
The last three precordial leads ($\text{V}_{\text{4--6}}$) are positive, as the
bulk of the electrical activity is towards them as the atrium depolarises from
right to left.

Pacing from site A, the P-Wave is positive in leads I, aVL and
$\text{V}_{\text{2--6}}$, it is negative in leads III, aVR and aVF and biphasic
in leads II and $\text{V}_{\text{1}}$.
The largest positive potential is observed in lead aVL, \mv{0.207}\ and the
largest negative potential is observed in lead III, \mv{-0.265}.
The $\text{T}_{\text{P}}$ waves are clearly visible in all leads except lead II
and $\text{V}_{\text{1}}$.
Lead $\text{V}_{\text{1}}$ is initially positive, as the excitation travels up
the crista terminalis towards the sinus node, before it becomes negative in the
latter half of the P-wave, as the left atrium starts to depolarise down from the
Bachmann's bundle.
Because the bulk direction of repolarisation is still from right to left, the
final three precordial ($\text{V}_{\text{4--6}}$) leads are still positive.


Pacing from site B, the P-Wave is positive in leads I, aVL and
$\text{V}_{\text{2--6}}$, it is negative in leads III, aVR and aVF and biphasic
in leads II and $\text{V}_{\text{1}}$.
The largest positive potential is observed in lead aVL, \mv{0.236}\ and the
largest negative potential is observed in lead III, \mv{-0.295}.
The $\text{T}_{\text{P}}$ waves are clearly visible in all leads except lead II
and $\text{V}_{\text{1}}$.
Lead $\text{V}_{\text{1}}$ is initially positive, as the excitation travels up
the crista terminalis towards the sinus node, before it becomes negative in the
latter half of the P-wave, as the left atrium starts to depolarise down from the
Bachmann's bundle.
Because the bulk direction of repolarisation is still from right to left, the
final three precordial ($\text{V}_{\text{4--6}}$) leads are still positive.








\subsection{Limitations}

Edenbrandt and Pahlm verses Uijen, optimised sets.  Perhaps Hyttinen paper.
Optimised for ST segment, not P-wave.
The original inverse dower matrix, as presented by Edenbrandt and
Pahlm~\cite{Edenbrandt1988}\ was used to perform the transformations, as used in
previous studies~\cite{Carlson2005,Holmqvist2007,Havmoller2007}

\section{Focal Atrial Tachycardia}

Atrial Tachycardias are one of the rarer forms of supraventricular
tachycardia, accounting 
They tend to occur as a result of other cardiac or respiratory diseases.
They are characterised by a high heart rate ($\geq$ \unit{250}{bpm}) and
typically have evidence of an abnormal cardiac axis or P-wave morphology.
They are hard to treat with drugs, but radiofrequency ablation can be used with
a high probability of success.
Diagnosis of atrial tachycardia can be difficult due to a lack of data.
Attempts to locate the sites of the ectopic focus are current topics of clinical
research~\cite{Kistler2006,Kahn2006,Yamane2001}.
This study shows how the model can provide more data for clinicians to study.

\subsection{Model of Focal Atrial Tachycardia}

To model focal atrial tachycardia, the model developed in the previous chapter
is used.
Instead of pacing from the cells corresponding to the sinus node, various sites
around the atria are selected.
These sites are shown in Figure~\ref{fig:forward:sites}.
All the nodes which have active cells within 10 cells (\mm{3.3}) are excited via
direct current injection of \unit{2}{nS}\ for \ms{2}.
The resulting excitation wave is then allowed to propagate without interference
and the BSPM and ECG calculated.
The ECGs are classified by the P-wave morphology and the time of first
deflection in each lead.



